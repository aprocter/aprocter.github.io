%%%%%%%%%%%%%%%%%%%%%%%%%%%%%%%%%%%%%%%%%%%%%%%%%%%%%%%%%%%%%%%%%%%%%%%%
%%%%%%%%%%%%%%%%%%%%%% Simple LaTeX CV Template %%%%%%%%%%%%%%%%%%%%%%%%
%%%%%%%%%%%%%%%%%%%%%%%%%%%%%%%%%%%%%%%%%%%%%%%%%%%%%%%%%%%%%%%%%%%%%%%%

%%%%%%%%%%%%%%%%%%%%%%%%%%%%%%%%%%%%%%%%%%%%%%%%%%%%%%%%%%%%%%%%%%%%%%%%
%% NOTE: If you find that it says                                     %%
%%                                                                    %%
%%                           1 of ??                                  %%
%%                                                                    %%
%% at the bottom of your first page, this means that the AUX file     %%
%% was not available when you ran LaTeX on this source. Simply RERUN  %%
%% LaTeX to get the ``??'' replaced with the number of the last page  %%
%% of the document. The AUX file will be generated on the first run   %%
%% of LaTeX and used on the second run to fill in all of the          %%
%% references.                                                        %%
%%%%%%%%%%%%%%%%%%%%%%%%%%%%%%%%%%%%%%%%%%%%%%%%%%%%%%%%%%%%%%%%%%%%%%%%

%%%%%%%%%%%%%%%%%%%%%%%%%%%% Document Setup %%%%%%%%%%%%%%%%%%%%%%%%%%%%

% Don't like 10pt? Try 11pt or 12pt
\documentclass[10pt]{article}

% This is a helpful package that puts math inside length specifications
\usepackage{calc}

% Simpler bibsection for CV sections
% (thanks to natbib for inspiration)
\makeatletter
\newlength{\bibhang}
\setlength{\bibhang}{1em}
\newlength{\bibsep}
 {\@listi \global\bibsep\itemsep \global\advance\bibsep by\parsep}
\newenvironment{bibsection}
    {\minipage[t]{\linewidth}\list{}{%
        \setlength{\leftmargin}{\bibhang}%
        \setlength{\itemindent}{-\leftmargin}%
        \setlength{\itemsep}{\bibsep}%
        \setlength{\parsep}{\z@}%
        }}
    {\endlist\endminipage}
\makeatother

% Layout: Puts the section titles on left side of page
\reversemarginpar

%
%         PAPER SIZE, PAGE NUMBER, AND DOCUMENT LAYOUT NOTES:
%
% The next \usepackage line changes the layout for CV style section
% headings as marginal notes. It also sets up the paper size as either
% letter or A4. By default, letter was used. If A4 paper is desired,
% comment out the letterpaper lines and uncomment the a4paper lines.
%
% As you can see, the margin widths and section title widths can be
% easily adjusted.
%
% ALSO: Notice that the includefoot option can be commented OUT in order
% to put the PAGE NUMBER *IN* the bottom margin. This will make the
% effective text area larger.
%
% IF YOU WISH TO REMOVE THE ``of LASTPAGE'' next to each page number,
% see the note about the +LP and -LP lines below. Comment out the +LP
% and uncomment the -LP.
%
% IF YOU WISH TO REMOVE PAGE NUMBERS, be sure that the includefoot line
% is uncommented and ALSO uncomment the \pagestyle{empty} a few lines
% below.
%

%% Use these lines for letter-sized paper
\usepackage[paper=letterpaper,
            %includefoot, % Uncomment to put page number above margin
            marginparwidth=1.2in,     % Length of section titles
            marginparsep=.05in,       % Space between titles and text
            margin=1in,               % 1 inch margins
            includemp]{geometry}

%% Use these lines for A4-sized paper
%\usepackage[paper=a4paper,
%            %includefoot, % Uncomment to put page number above margin
%            marginparwidth=30.5mm,    % Length of section titles
%            marginparsep=1.5mm,       % Space between titles and text
%            margin=25mm,              % 25mm margins
%            includemp]{geometry}

%% More layout: Get rid of indenting throughout entire document
\setlength{\parindent}{0in}

%% This gives us fun enumeration environments. compactitem will be nice.
\usepackage{paralist}

%% Reference the last page in the page number
%
% NOTE: comment the +LP line and uncomment the -LP line to have page
%       numbers without the ``of ##'' last page reference)
%
% NOTE: uncomment the \pagestyle{empty} line to get rid of all page
%       numbers (make sure includefoot is commented out above)
%
\usepackage{fancyhdr,lastpage}
\pagestyle{fancy}
%\pagestyle{empty}      % Uncomment this to get rid of page numbers
\fancyhf{}\renewcommand{\headrulewidth}{0pt}
\fancyfootoffset{\marginparsep+\marginparwidth}
\newlength{\footpageshift}
\setlength{\footpageshift}
          {0.5\textwidth+0.5\marginparsep+0.5\marginparwidth-2in}
\lfoot{\hspace{\footpageshift}%
       \parbox{4in}{\, \hfill %
                    \arabic{page} of \protect\pageref*{LastPage} % +LP
%                    \arabic{page}                               % -LP
                    \hfill \,}}

% Finally, give us PDF bookmarks
\usepackage{color,hyperref}
\definecolor{darkblue}{rgb}{0.0,0.0,0.3}
\hypersetup{colorlinks,breaklinks,
            linkcolor=darkblue,urlcolor=darkblue,
            anchorcolor=darkblue,citecolor=darkblue}

%%%%%%%%%%%%%%%%%%%%%%%% End Document Setup %%%%%%%%%%%%%%%%%%%%%%%%%%%%


%%%%%%%%%%%%%%%%%%%%%%%%%%% Helper Commands %%%%%%%%%%%%%%%%%%%%%%%%%%%%

% The title (name) with a horizontal rule under it
%
% Usage: \makeheading{name}
%
% Place at top of document. It should be the first thing.
\newcommand{\makeheading}[1]%
        {\hspace*{-\marginparsep minus \marginparwidth}%
         \begin{minipage}[t]{\textwidth+\marginparwidth+\marginparsep}%
                {\large \bfseries #1}\\[-0.15\baselineskip]%
                 \rule{\columnwidth}{1pt}%
         \end{minipage}}

% The section headings
%
% Usage: \section{section name}
%
% Follow this section IMMEDIATELY with the first line of the section
% text. Do not put whitespace in between. That is, do this:
%
%       \section{My Information}
%       Here is my information.
%
% and NOT this:
%
%       \section{My Information}
%
%       Here is my information.
%
% Otherwise the top of the section header will not line up with the top
% of the section. Of course, using a single comment character (%) on
% empty lines allows for the function of the first example with the
% readability of the second example.
\renewcommand{\section}[2]%
        {\pagebreak[2]\vspace{1.3\baselineskip}%
         \phantomsection\addcontentsline{toc}{section}{#1}%
         \hspace{0in}%
         \marginpar{
         \raggedright \scshape #1}#2}

% An itemize-style list with lots of space between items
\newenvironment{outerlist}[1][\enskip\textbullet]%
        {\begin{itemize}[#1]}{\end{itemize}%
         \vspace{-.6\baselineskip}}

% An environment IDENTICAL to outerlist that has better pre-list spacing
% when used as the first thing in a \section
\newenvironment{lonelist}[1][\enskip\textbullet]%
        {\vspace{-\baselineskip}\begin{list}{#1}{%
        \setlength{\partopsep}{0pt}%
        \setlength{\topsep}{0pt}}}
        {\end{list}\vspace{-.6\baselineskip}}

% An itemize-style list with little space between items
\newenvironment{innerlist}[1][\enskip\textbullet]%
        {\begin{compactitem}[#1]}{\end{compactitem}}

% To add some paragraph space between lines.
% This also tells LaTeX to preferably break a page on one of these gaps
% if there is a needed pagebreak nearby.
\newcommand{\blankline}{\quad\pagebreak[2]}

% 

%%%%%%%%%%%%%%%%%%%%%%%% End Helper Commands %%%%%%%%%%%%%%%%%%%%%%%%%%%

%%%%%%%%%%%%%%%%%%%%%%%%% Begin CV Document %%%%%%%%%%%%%%%%%%%%%%%%%%%%

\begin{document}
\makeheading{Adam Procter, Ph.D.}

\section{Contact Information}
%
% NOTE: Mind where the & separators and \\ breaks are in the following
%       table.
%
% ALSO: \rcollength is the width of the right column of the table
%       (adjust it to your liking; default is 1.85in).
%
\newlength{\rcollength}\setlength{\rcollength}{2.1in}%
%
\begin{tabular}[t]{@{}p{\textwidth-\rcollength}p{\rcollength}}
\href{http://www.cs.missouri.edu/}%
     {Department of Computer Science} & \\
\href{http://www.missouri.edu/}{University of Missouri} & +1 573 819 2767 \\
201 Engineering Building West & \href{mailto:adamprocter@mail.missouri.edu}{adamprocter@mail.missouri.edu}\\
Columbia, MO  65211 USA       & \href{http://adamprocter.com/}{http://adamprocter.com}\\
\end{tabular}

%\section{Objective}
%
%To obtain a post-doctoral position in the academic, government, or private sector that draws on my demonstrated talent for cutting-edge research in programming languages, program verification, and information security.

\section{Research Interests}
%
Functional programming, semantics of programming languages, hardware synthesis from functional languages, hardware and software verification (especially for security), language-based security, model-driven implementation techniques for secure systems, computer-assisted theorem proving.

\section{Education}
%
\textbf{\href{http://www.missouri.edu/}{University of Missouri}},
Columbia, Missouri USA
\begin{outerlist}

\item[] Ph.D.,
        \href{http://www.cs.missouri.edu/}
             {Computer Science}, December 2014
        \begin{innerlist}
        \item Dissertation: {\em Semantics-Driven Design and Implementation of High-Assurance Hardware}
        \item Advisor:
              \href{http://www.missouri.edu/~harrisonwl/}
                   {William L. Harrison}
%        \item GPA: 3.96/4.00
        \end{innerlist}

\item[] B.A.,
        \href{http://www.cs.missouri.edu/}
             {Computer Science}, May 2005
        \begin{innerlist}
        \item Minor in \href{http://www.math.missouri.edu/}{Mathematics}
        \item Graduated \emph{summa cum laude}
%        \item GPA: 3.91/4.00
        \end{innerlist}

\end{outerlist}

\section{Conference Publications} \begin{bibsection}
    \item Ian Graves, Adam Procter, Wililam L. Harrison, Michela Becchi, and Gerard Allwein. Provably Correct Development of Reconfigurable Hardware Designs via Equational Reasoning.
	\emph{Proceedings of the 2015 International Conference on Field-Programmable Technology (ICFPT'15)}, Queenstown, New Zealand, December 2015.

    \item Adam Procter, William L. Harrison, Ian Graves, Michela Becchi, and Gerard Allwein. Semantics Driven Hardware Design, Implementation, and Verification with ReWire.
        \emph{Proceedings of the 2015 ACM SIGPLAN/SIGBED Conference on Languages, Compilers, Tools and Theories for Embedded Systems (LCTES'15)}, Portland, June 2015.

    \item Ian Graves, Adam Procter, William L. Harrison, Michela Becchi, and Gerard Allwein. Hardware Synthesis from Functional Embedded Domain-Specific Languages: A Case Study in Regular Expression Compilation.
        \emph{Proceedings of the 11th International Symposium on Applied Reconfigurable Computing (ARC'15)}, Bochum, April 2015.

    \item Adam Procter, William L. Harrison, Ian Graves, Michela Becchi, and Gerard Allwein. Semantics-directed Machine Architecture in ReWire.
        \emph{Proceedings of the 2013 International Conference on Field-Programmable Technology (ICFPT'13)}, Kyoto, December 2013.

    \item William L. Harrison, Adam Procter, and Gerard Allwein. The Confinement Problem in the Presence of Faults.
        \emph{Proceedings of the 14th International Conference on Formal Engineering Methods (ICFEM'12)}, Kyoto, November 2012.

    \item Chris Hathhorn, Michela Becchi, William L. Harrison and Adam Procter. Formal semantics of heterogeneous CUDA-C: A modular approach with applications.
        \emph{Proceedings of the 2012 Systems Software Verification Conference (SSV'12)}, Sydney, November 2012.

    \item Adam Procter, William L. Harrison, and Aaron Stump. The Design of a Practical Theorem Prover for a Lazy Functional Language.
        \emph{Proceedings of the 2012 Symposium on Trends in Functional Programming (TFP'12)}, St Andrews, UK, June 2012.

\end{bibsection}
\begin{bibsection}

    \item Michela Becchi, Kittisak Sajjapongse, Ian Graves, Adam Procter, Vignesh Ravi, and Srimat Chakradhar.
            A Virtual Memory Based Runtime to Support Multitenancy in Clusters with GPUs. \emph{Proceedings of the 21st International Symposium on High-Performance Parallel and Distributed Computing (HPDC'12)}, Delft, June 2012. {\bf (Best paper award!)}

    \item William L. Harrison, Benjamin Schulz, Adam Procter, Andrew Lukefahr, and Gerard Allwein.
        Towards Semantics-directed System Design and Synthesis. Invited
        paper. \emph{Proceedings of the 2011 International Conference on Engineering of Reconfigurable Systems and Algorithms (ERSA'11)}, Las Vegas, July 2011.

    \item William L. Harrison, Adam M. Procter, Jason Agron, Garrin Kimmell, and Gerard Allwein.
        Model-driven Engineering from Modular Monadic Semantics: Implementation
        Techniques Targeting Hardware and Software. \emph{Proceedings of the IFIP
        Working Conference on Domain Specific Languages (DSLWC)}, Oxford, July 2009.

    \item Pericles S. Kariotis, Adam M. Procter, and William L. Harrison. Making
        Monads First-class with Template Haskell. \emph{Proceedings of the ACM
        SIGPLAN 2008 Haskell Symposium (Haskell '08)}, Victoria, BC, Canada, September 2008.

    \item William L. Harrison, Gerard Allwein, Andy Gill, and Adam Procter.
        Asynchronous Exceptions as an Effect. \emph{Proceedings of the Ninth
        International Conference on Mathematics of Program Construction (MPC'08)},
        Marseille, July 2008.
\end{bibsection}

\section{Journal Publications} \begin{bibsection}
    \item William L. Harrison and Adam M. Procter. Cheap (But Functional) Threads.
        Submitted to \emph{Journal of Functional Programming},
        draft available by request.
\end{bibsection}

\section{Professional Service}
%
\vspace{-1em}
\begin{innerlist}
\item External reviewer for IFL'11.
\item Helped organize Midwest Verification Day 2014 in Columbia, MO.
\end{innerlist}

\section{Teaching Experience}
%
\textbf{\href{http://www.cs.missouri.edu/}{Department of Computer Science}, University of Missouri}
\begin{innerlist}
\item Instructor, Principles of Programming Languages, Fall 2010 and Fall 2012
\item Teaching Assistant, Principles of Programming Languages, Fall 2009, Spring 2008, and Spring 2007
\item Teaching Assistant, Production Languages (Programming in C), Fall 2007 and Fall 2006
\item Instructor, Production Languages (Programming in C), Spring 2007
\item Instructor, Algorithm Design and Programming I, Spring 2006
\item Teaching Assistant, Algorithm Design and Programming I, Fall 2005
\end{innerlist}

\blankline

\textbf{\href{http://japanesestudies.missouri.edu/}{Japanese Studies Program}, University of Missouri}
\begin{innerlist}
\item Instructor, Elementary Japanese II, Spring 2005
\item Teaching Assistant, Elementary Japanese I, Fall 2004
\end{innerlist}

\blankline

\textbf{\href{http://success.missouri.edu/tlc.html}{The Learning Center}, University of Missouri}
\begin{innerlist}
\item Tutor, Summer 2002---Spring 2004
\end{innerlist}

\section{Employment}
%
\textbf{\href{http://chaco.missouri.edu/}{Center for High Assurance Computing}, University of Missouri}
\begin{outerlist}
\item[] \textit{Postdoctoral Fellow} \hfill November 2014---
        \begin{innerlist}
        \item Postdoctoral researcher at the \href{http://chaco.missouri.edu/}{Center for High Assurance Computing}.
        \item Supervisor: Professor William L. Harrison.
        \end{innerlist}
\end{outerlist}

\blankline

\textbf{\href{http://www.cs.missouri.edu/}{Department of Computer Science}, University of Missouri}
\begin{outerlist}
\item[] \textit{Graduate Research Assistant} \hfill June 2008---December 2010
        \begin{innerlist}
        \item Research assistant at the \href{http://hask.cs.missouri.edu/}{High Assurance Security Kernel (HASK) Lab}, under Professor William L. Harrison.
        \end{innerlist}

\item[] \textit{Graduate Teaching Assistant} \hfill August 2005---May 2008\\
        \color{white}.\color{black}          \hfill August 2009---December 2009\\
        \color{white}.\color{black}          \hfill August 2010---December 2010
        \begin{innerlist}
        \item Served as instructor or teaching assistant for several diferent computer science courses.
        \item Duties ranged from grading and holding office hours to teaching a large lecture course and supervising four teaching assistants.
        \end{innerlist}
\end{outerlist}

\blankline

\textbf{\href{http://www.cs.uiowa.edu/}{Department of Computer Science}, University of Iowa}
\begin{outerlist}
\item[] \textit{Short-Term Scientific Employee (Summer Visitor)} \hfill June 2010---August 2010
        \begin{innerlist}
        \item Developed a theorem-proving system for monadic programs in collaboration with Professor Aaron Stump.
        \end{innerlist}
\end{outerlist}

\blankline

\textbf{\href{http://www.biology.missouri.edu/}{Division of Biological Sciences}, University of Missouri}
\begin{outerlist}
\item[] \textit{Computer Programmer} \hfill June 2006---May 2009
        \begin{innerlist}
        \item Developed a LabVIEW-based application to play back aural stimuli to insects in support of behavioral and neurological experiments.
        \end{innerlist}
\end{outerlist}

\blankline

\textbf{\href{http://japanesestudies.missouri.edu/}{Japanese Studies Program}, University of Missouri}
\begin{outerlist}
\item[] \textit{Peer Learning Assistant} \hfill August 2004---May 2005
        \begin{innerlist}
        \item Served as co-instructor for Elementary Japanese II in Spring 2005.
        \item Conducted two weekly lab sessions for Elementary Japanese I in Fall 2004.
        \end{innerlist}
\end{outerlist}

\blankline

\textbf{\href{http://success.missouri.edu/tlc.html}{The Learning Center}, University of Missouri}
\begin{outerlist}
\item[] \textit{Tutor} \hfill June 2002---May 2004
        \begin{innerlist}
        \item Tutored individual students in computer science and elementary Japanese.
        \item Held group tutoring sessions in computer science.
        \end{innerlist}
\end{outerlist}

\section{Honors and Awards}
%
Fellowships and Scholarships
\begin{innerlist}
\item Graduate Assistance in Areas of National Need (GAANN) Fellowship, 2011---2014
\item Gilliom Graduate Fellowship in Cyber Security, 2007---2009
\item State of Missouri Bright Flight Scholarship, 2000---2005
\end{innerlist}

\blankline

Honors
\begin{innerlist}
\item Honorary student marshal, University of Missouri College of Arts and Science commencement ceremony, May 2005
\item University of Missouri College of Arts and Science dean's list every semester, Fall 2000---Spring 2005
\item Phi Beta Kappa (junior-year inductee, 2003)
\end{innerlist}

\section{Technical Skills}
%
Programming Languages
\begin{innerlist}
\item Haskell, ML, Coq, C, Java, Perl, UNIX shell scripting, Python, Ruby, PHP, JavaScript, Visual Basic, LabVIEW
\end{innerlist}

\blankline

Hardware Description Languages
\begin{innerlist}
\item VHDL
\end{innerlist}

\blankline

Operating Systems
\begin{innerlist}
\item UNIX/Linux, Microsoft Windows, Mac OS X
\end{innerlist}

\blankline

Other
\begin{innerlist}
\item HTML, CSS, SQL, MySQL and Oracle databases, Linux system administration
\end{innerlist}

%\section{Activities and Volunteer Work}
%
%Volunteer system administration for Mizzou Linux Users Group

%\blankline

%Miscellaneous computer support tasks for \href{http://www.gromo.org/}{GRO -- Grass Roots Organizing}, a not-for-profit social justice organization

\section{Languages}
%
English (native speaker)

Japanese (read, write, and speak at a high intermediate to advanced level)

German (once intermediate, now quite rusty)

\section{References}
%
Available upon request.

\end{document}

%%%%%%%%%%%%%%%%%%%%%%%%%% End CV Document %%%%%%%%%%%%%%%%%%%%%%%%%%%%%
